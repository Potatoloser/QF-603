%%%%%%%%%%%%%%%%%%%%%%%%%%%%%%%%%%%%%%%%%%%%%%%%%%%%%%%%%%%%%%%%%%%%%%%%%%%%%%%%
%%%%%%%%%%%%%%%%%%%%%%%%%%%%%%%%%%%%%%%%%%%%%%%%%%%%%%%%%%%%%%%%%%%%%%%%%%%%%%%%
%%% Template for AIMS Rwanda Assignments         %%%              %%%
%%% Author:   AIMS Rwanda tutors                             %%%   ###        %%%
%%% Email: tutors2017-18@aims.ac.rw                               %%%   ###        %%%
%%% Copyright: This template was designed to be used for    %%% #######      %%%
%%% the assignments at AIMS Rwanda during the academic year %%%   ###        %%%
%%% 2017-2018.                                              %%%   #########  %%%
%%% You are free to alter any part of this document for     %%%   ###   ###  %%%
%%% yourself and for distribution.                          %%%   ###   ###  %%%
%%%                                                         %%%              %%%
%%%%%%%%%%%%%%%%%%%%%%%%%%%%%%%%%%%%%%%%%%%%%%%%%%%%%%%%%%%%%%%%%%%%%%%%%%%%%%%%
%%%%%%%%%%%%%%%%%%%%%%%%%%%%%%%%%%%%%%%%%%%%%%%%%%%%%%%%%%%%%%%%%%%%%%%%%%%%%%%%


%%%%%% Ensure that you do not write the questions before each of the solutions because it is not necessary. %%%%%% 

\documentclass[12pt,a4paper]{article}

%%%%%%%%%%%%%%%%%%%%%%%%% packages %%%%%%%%%%%%%%%%%%%%%%%%
\usepackage{graphicx}
\usepackage{tabulary}

\usepackage{amsmath}
\usepackage{fancyhdr}
\usepackage{amssymb}
\usepackage{amsthm}
\usepackage{placeins}
\usepackage{amsfonts}
\usepackage{graphicx}
\usepackage[all]{xy}
\usepackage{tikz}
\usepackage{verbatim}
\usepackage[left=2cm,right=2cm,top=3cm,bottom=2.5cm]{geometry}
\usepackage{hyperref}
\usepackage{caption}
\usepackage{subcaption}
\usepackage{multirow}
\usepackage{psfrag}



%%%%%%%%%%%%%%%%%%%%% students data %%%%%%%%%%%%%%%%%%%%%%%%

\newcommand{\student}{\textbf{Zhou Shen}}
\newcommand{\course}{\textbf{QF603 Quantitative Analysis of Financial Market}}
\newcommand{\assignment}{\textbf{2}}

%%%%%%%%%%%%%%%%%%% using theorem style %%%%%%%%%%%%%%%%%%%%
\newtheorem{thm}{Theorem}
\newtheorem{lem}[thm]{Lemma}
\newtheorem{defn}[thm]{Definition}
\newtheorem{exa}[thm]{Example}
\newtheorem{rem}[thm]{Remark}
\newtheorem{coro}[thm]{Corollary}
\newtheorem{quest}{Question}[section]
%%%%%%%%%%%%%%%%%%%%%%%%%%%%%%%%%%%%%%%%
\usepackage{lipsum}%% a garbage package you don't need except to create examples.
\usepackage{fancyhdr}
\usepackage{hyperref}
\usepackage{graphicx}
\usepackage{pythonhighlight}
\pagestyle{fancy}
%\lhead{Azamuke Denish}
\rhead{ \thepage}
%\cfoot{\textbf{AIMS Rwanda Academic Year 2020 - 2021}}
\renewcommand{\headrulewidth}{0.4pt}
\renewcommand{\footrulewidth}{0.4pt}

%%%%%%%%%%%%%%  Shortcut for usual set of numbers  %%%%%%%%%%%

\newcommand{\N}{\mathbb{N}}
\newcommand{\Z}{\mathbb{Z}}
\newcommand{\Q}{\mathbb{Q}}
\newcommand{\R}{\mathbb{R}}
\newcommand{\C}{\mathbb{C}}

%%%%%%%%%%%%%%%%%%%%%%%%%%%%%%%%%%%%%%%%%%%%%%%%%%%%%%%555
\begin{document}

%%%%%%%%%%%%%%%%%%%%%%% title page %%%%%%%%%%%%%%%%%%%%%%%%%%
\thispagestyle{empty}
\begin{center}
	\includegraphics[scale = 0.4]{D:/SMU/SMU-LKCSB-logo.png}
	%\textbf{AFRICAN INSTITUTE FOR MATHEMATICAL SCIENCES \\[0.5cm]
	%(AIMS RWANDA, KIGALI)}
	\vspace{0.5cm}
\end{center}
%%%%%%%%%%%%%%%%%%%%% assignment information %%%%%%%%%%%%%%%%
\noindent
\rule{17cm}{0.2cm}\\[0.3cm]
Name: \student \hfill Assignment Number: \assignment\\[0.1cm]
Course: \course \hfill Date: \today\\
\rule{17cm}{0.05cm}
\vspace{1.0cm}
%%%%%%%%%%%%%%%%%%%%%%%%%%%%%%%%%%%%%%%%%%%%%

\pdfbookmark[1]{Problem 1}{p1}
\textbf{Problem 1.}
Based on the Assumptions of OLS:
\[
\begin{aligned}
    E(e_{i}) &= 0\ \rm{for\ every}\ i \\
    E(e_{i}^2) &= \sigma_{e}^2  \\
    E(e_{i}e_{j}) &= 0\ \rm{for\ every}\ i,j \\
    X_{i},&e_{j}\ \rm{are\ independent\ for\ each}\ i,j \\
    e_{i}&\stackrel{d}{\sim}N(0,\sigma_{e}^2)
\end{aligned}
\]

The answer is 3.
\\

\pdfbookmark[2]{Problem 2}{p2}
\textbf{Problem 2.}

\[
\begin{aligned}
\rm{Annual\ Return\ Volatility} &= \rm{Daily\ Return\ Volatility} \times \sqrt{\rm{trade days}}\\
                        &= 0.5 \% \times \sqrt{252}\\
                        &= 7.93 \%
\end{aligned}
\]

So the answer is 1.\ 7.93\%.
\\

\pdfbookmark[3]{Problem 3}{p3}
\textbf{Problem 3.}

Ordered probit regression is for dependent variable which takes a number of infinite and discrete values that contain ordinal information.

So the answer is 2.\ Ordered probit regression.
\\

\pdfbookmark[4]{Problem 4}{p4}
\textbf{Problem 4.}

Total number of heads follow the Binomial Distribution. According to

\[
Var(X) = np(1-p)
\]

So the answer is $100p(1-p)$.
\\

\pdfbookmark[5]{Problem 5}{p5}
\textbf{Problem 5.}

According to the concept of Uniform Distribution, we have probability distribution function of $x$:

\[
f(x)=
\begin{cases}

\frac{1}{5}       & 5\leq x \leq10 \\
0  &  \rm{other} 

\end{cases}
\]

So we can calculate variance of X by

\[
\begin{aligned}
E(X) &= \int_{X_1}^{X_2}f(x)x dx = \int_{5}^{10}\frac{1}{5}x dx\\
&=7.5\\
E(X^2) &= \int_{X_1}^{X_2}f(x)x^2 dx = \int_{5}^{10}\frac{1}{5}x^2 dx\\
&=\frac{175}{3}\\
Var(X) &= E(X^2) - E(X)^2 = \frac{175}{3} - 7.5^2\\
&=\frac{25}{12}
\end{aligned}
\]

So the variance of x is $\frac{25}{12}$.
\\

\pdfbookmark[6]{Problem 6}{p6}
\textbf{Problem 6.}

The cumulative distribution function of defaults in portfolio is 

\[
\begin{aligned}
\mathbb{P}(X = n) &= \frac{\lambda^n}{n!}e^{-\lambda}\\
&=\frac{10^n}{n!}e^{-10}
\end{aligned}
\]

So in one year, the probability that there are exactly 2 defaults is 

\[
\begin{aligned}
\mathbb{P}(X = 2) &= \frac{10^2}{2!}e^{-10}\\
&= 50e^{-10}\\
&\approx 0.23 \%
\end{aligned}
\]

In two years, the probability that there are exactly 2 defaults is 

\[
\begin{aligned}
\mathbb{P}(X = 2) &= \frac{20^2}{2!}e^{-20}\\
&= 200e^{-20}\\
&\approx 4.12\times 10^{-5} \%
\end{aligned}
\]



\end{document}
